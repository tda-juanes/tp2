\section{Introducci\'on}

En el presente trabajo pr\'actico se realiza un analisis de un algoritmo por
\textit{Programaci\'on Din\'amica} para resolver un problema de maximizaci\'on.

\section{Consideraciones}

Un cronograma de entrenamiento indica cuanto esfuerzo demanda el entrenamiento
de cada d\'ia. Se puede calcular la ganancia de cada d\'ia como el m\'inimo
entre el esfuerzo requerido por el entrenamiento para ese d\'ia $\left( e_i
\right)$, y la eneg\'ia que tienen los jugadores en ese d\'ia $\left( j
\right)$ o $0$ si descansan.

La energ\'ia que tienen los jugadores en un d\'ia determinado depende de hace
cuantos d\'ias descansaron por \'ultima vez y est\'a dada por la secuencia
$\mathcal{S}$:

\begin{equation*}
    \mathcal{S} = s_1, s_2, ..., s_n
\end{equation*}
\begin{equation*}
    \text{con } s_1 \ge s_2 \ge ... \ge s_n
\end{equation*}

\section{Problema}

Dada la secuencia de energ\'ia disponible desde el \'ultimo descanso
$\mathcal{S}$, y el esfuerzo/ganancia de cada d\'ia, determinar la m\'axima
cantidad de ganancia que se puede obtener de los entrenamientos, considerando
posibles descansos.

\subsection{Ecuaci\'on de Recurrencia}

Los jugadores comienzan descansados y sin ninguna ganancia:

\begin{equation*}
    \mathcal{G} \left( 0, 0 \right) = 0
\end{equation*}

Si eligen descansar en el d\'ia $k$, al d\'ia siguiente se renueva su
energ\'ia, por lo que lo mejor ser\'a maximizar la ganancia del d\'ia anterior:

\begin{equation}
\label{eq:g_{0k}}
    \mathcal{G} \left( 0, k \right) = \max_{0 \le i < k} \left( \mathcal{G} \left( i, k - 1 \right) \right)
\end{equation}

Por otro lado, si no descansan la ganancia para ese d\'ia depende de su
energ\'ia y de la ganancia del d\'ia anterior:

\begin{equation*}
    \mathcal{G} \left( i, k \right) = \mathcal{G} \left( i - 1, k - 1 \right) + \min \left( s_i, e_k \right)
\end{equation*}

Lo que nos deja con la siguiente ecuaci\'on de recurrencia:

\begin{equation}
    \begin{cases}
        \mathcal{G} \left( 0, 0 \right) = 0\\
        \mathcal{G} \left( 0, k \right) = \max_{0 \le i < k} \left( \mathcal{G} \left( i, k - 1 \right) \right)\\
        \mathcal{G} \left( i, k \right) = \mathcal{G} \left( i - 1, k - 1 \right) + \min \left( s_i, e_k \right)\\
    \end{cases}
\end{equation}

donde $\mathcal{G} \left( 0, n + 1 \right)$ es la soluci\'on al problema.

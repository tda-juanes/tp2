\subsection{Algoritmo}

Para encontrar $\mathcal{G} \left( 0, n + 1 \right)$, proponemos el siguiente
algoritmo por \textit{Programaci\'on Din\'amica}:

Para cada d\'ia, considerar la posibilidad de haber descansado por \'ultima
vez en cualquiera de los d\'ias anteriores, y calcular la ganancia
correspondiente. Adem\'as, considerar que se puede descansar en el mismo d\'ia,
en cuyo caso la ganancia m\'axima del d\'ia anterior es la ganancia para ese
d\'ia.

De esta forma podemos calcular las posibles ganancias de cada d\'ia de
entrenamiento utilizando \'unicamente las posibles ganancias del d\'ia
anterior. Como pudieron haber descansado por \'ultima vez en cualquiera de los
d\'ias anteriores, los cuales son a lo sumo $n$, la complejidad espacial de
nuestro algoritmo es $\mathcal{O}(n)$.

\subsection{Implementaci\'on}
\label{sec:implementacion}

\lstinputlisting[language=Python]{code/solution.py}

La ecuaci\'on de recurrencia que corresponde a este algoritmo es:
\begin{equation*}
    \mathcal{T}(n) = \mathcal{T}\left(n - 1\right) + \mathcal{O}\left(n\right)
\end{equation*}

Esto es porque en cada iteraci\'on calculamos el m\'aximo de un arreglo de
tama\~no $n$ $\left( \mathcal{O}(n) \right)$, e iteramos por el mismo
realizando operaciones de tiempo constante $\left( \mathcal{O}(n) \right)$.

Aplicando el teorema maestro, o multiplicando por la cantidad de
entrenamientos, nos queda que la complejidad es $\mathcal{O}(n^2)$.
 
\subsection{Reconstrucci\'on}

Para encontrar el camino que maximiza el esfuerzo alcanza con tener el estado final
del arreglo de ganancias propuesto \hyperref[sec:implementacion]{anteriormente},
ya que con este se pueden reconstruir todos los estados anteriores del mismo.
Considerando un arreglo de ganancias para el día $n$ con un máximo en la
posici\'on $k$ se da que:
\begin{equation*}
    ganancias[k] = \mathcal{G} ( 0, k + 1 ) + \sum_{i=1}^{n-k} \min (s_i, e_{k+i})
\end{equation*}
    Que es equivalente a decir que el camino cumple con haber descanzado el día
$k$ y trabajado todos los días posteriores.\\

    Con esto en mente, alcanza con restarle la sumatoria 
$\sum_{i=1}^{n-k} \min (s_i, e_{k+i})$ al elemento en la posici\'on $k$ para
recuperar el valor que ten\'ia en el d\'ia $k$.\\
Generalizando: el valor en el \'indice $j$ para el d\'ia $k$, con $j \leq k$,
puede recuperarse con:
\begin{equation*}
    ganancias[j] - \sum_{i=1}^{n-k} \min (s_{k+i-j}, e_{k+i})
\end{equation*}

    Una vez reconstruido el arreglo para el d\'ia $k$, y considerando la
ecuaci\'on para $\mathcal{G} ( 0, k )$ \eqref{eq:g_{0k}}, tenemos que el problema
se reduce a una versi\'on m\'as peque\~na de s\'i mismo. 

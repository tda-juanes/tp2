\section{Conclusiones}

La soluci\'on propuesta plantea la utilización del día anterior para el calculo de un entrenamiento óptimo. 
Como solo prescindimos del entrenamiento del d\'ia anterior para calcular nuestro d\'ia actual, la \textit{memoization} requerida es de espacio lineal.

Al haber planteado un algoritmo de complejidad exponencial cuadratica, no podemos decir que la soluci\'on es veloz, pero sin embargo es la m\'as \'optima posible, el m\'etodo de reconstrucci\'on por su parte s\'i es r\'apido, ya que tiene complejidad lineal.

Destacamos que nuestro algorimo no considera la condicion inicial de $e_i >= e_{i+1}$, por lo que nuestro programa podria hayar soluciones para casos en los que no este dada esta restricci\'on.

\section{Conclusiones}

La Programaci\'on Din\'amica demostr\'o ser una t\'ecnica eficiente para resolver el problema de Scaloni,
reduciendo a $\mathcal{O}(n^2)$ un problema que parec\'ia tener complejidad temporal exponencial.

No solo eso, sino que tambi\'en pudimos reconstruir la soluci\'on en $\mathcal{O}(n^2)$
-por lo tanto no empeorando la complejidad del algor\'itmo- y todo esto manteniendo la complejidad espacial $\mathcal{O}(n)$.

Destacamos que nuestro algorimo no considera la condicion inicial de $e_i >= e_{i+1}$, por lo que nuestro programa podria hayar soluciones para casos en los que no este dada esta restricci\'on.
